The advent of smart homes has only just begun and has already created new threats. While more connected devices bring increased insights and efficiencies to their users, these networked systems also become more vulnerable and may allow adversaries to attack huge swathes of the population simultaneously. These attacks are not limited to data breaches, as is the case with most software-only security problems today, but can also put the user’s property in danger. Hence, necessary measures to protect consumer security and privacy should be taken by the involved parties.

Professionals who standardize communications should recognize that not every developer has security expertise that will let them comply with written law in an international market, let alone make a device that is actually secure. Certain measures are needed to ensure that developers are being responsible without sacrificing from the innovative capability of new applications. While there are some existing measures, both in the technical and the policy landscape, to improve ease of integration and protect the privacy and security of IoT device users, more can be done to improve the current measures to adapt to the challenges of newer technologies.

Communications protocols can adopt certain additional measures (listed above) in a way that provides simple, secure interfaces to developers who want to add communications to physical devices. We believe these measures will provide higher security to easy-to-implement solutions, which are mostly the settings used for IoT implementation in smart homes. 
