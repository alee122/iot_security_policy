\begin{mdframed}[backgroundcolor=gray!10]
Networked devices bring benefits that a non-networked device is incapable of providing, such as:

\begin{description}
\item[EcoBee4:] a homeowner is concerned about pipes bursting in the cold and so sets a remote sensor near the vulnerable pipes and programs the thermostat to keep that part of the house at a safe temperature no matter what (and no matter where the thermostat itself is situated).
\item[August Smart Lock Pro:]  a smart locks is a valuable tool for an owner of an AirBnB—they can grant access to a stranger for precisely the time the property is rented without meeting in person to exchange keys.
\item[DAHUA 4MP IR WiFi 2.8mm Mini Bullet:] installing a network of surveillance cameras is much easier if wires are not needed to carry data, and feeds can be easily checked while not at home.
\end{description}
Applying the network topology models is difficult, given that most smart home devices are designed to allow the customer to do piecemeal rollouts in their own homes instead of buying a whole ecosystem at once. The Ecobee and its sensors is an example of a “star” (also commonly called “hub-and-spoke”) network model. As the average person owns more devices networks topology is likely to diversify.
\end{mdframed}
